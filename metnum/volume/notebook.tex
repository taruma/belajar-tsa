
% Default to the notebook output style

    


% Inherit from the specified cell style.




    
\documentclass[11pt]{article}

    
    
    \usepackage[T1]{fontenc}
    % Nicer default font (+ math font) than Computer Modern for most use cases
    \usepackage{mathpazo}

    % Basic figure setup, for now with no caption control since it's done
    % automatically by Pandoc (which extracts ![](path) syntax from Markdown).
    \usepackage{graphicx}
    % We will generate all images so they have a width \maxwidth. This means
    % that they will get their normal width if they fit onto the page, but
    % are scaled down if they would overflow the margins.
    \makeatletter
    \def\maxwidth{\ifdim\Gin@nat@width>\linewidth\linewidth
    \else\Gin@nat@width\fi}
    \makeatother
    \let\Oldincludegraphics\includegraphics
    % Set max figure width to be 80% of text width, for now hardcoded.
    \renewcommand{\includegraphics}[1]{\Oldincludegraphics[width=.8\maxwidth]{#1}}
    % Ensure that by default, figures have no caption (until we provide a
    % proper Figure object with a Caption API and a way to capture that
    % in the conversion process - todo).
    \usepackage{caption}
    \DeclareCaptionLabelFormat{nolabel}{}
    \captionsetup{labelformat=nolabel}

    \usepackage{adjustbox} % Used to constrain images to a maximum size 
    \usepackage{xcolor} % Allow colors to be defined
    \usepackage{enumerate} % Needed for markdown enumerations to work
    \usepackage{geometry} % Used to adjust the document margins
    \usepackage{amsmath} % Equations
    \usepackage{amssymb} % Equations
    \usepackage{textcomp} % defines textquotesingle
    % Hack from http://tex.stackexchange.com/a/47451/13684:
    \AtBeginDocument{%
        \def\PYZsq{\textquotesingle}% Upright quotes in Pygmentized code
    }
    \usepackage{upquote} % Upright quotes for verbatim code
    \usepackage{eurosym} % defines \euro
    \usepackage[mathletters]{ucs} % Extended unicode (utf-8) support
    \usepackage[utf8x]{inputenc} % Allow utf-8 characters in the tex document
    \usepackage{fancyvrb} % verbatim replacement that allows latex
    \usepackage{grffile} % extends the file name processing of package graphics 
                         % to support a larger range 
    % The hyperref package gives us a pdf with properly built
    % internal navigation ('pdf bookmarks' for the table of contents,
    % internal cross-reference links, web links for URLs, etc.)
    \usepackage{hyperref}
    \usepackage{longtable} % longtable support required by pandoc >1.10
    \usepackage{booktabs}  % table support for pandoc > 1.12.2
    \usepackage[inline]{enumitem} % IRkernel/repr support (it uses the enumerate* environment)
    \usepackage[normalem]{ulem} % ulem is needed to support strikethroughs (\sout)
                                % normalem makes italics be italics, not underlines
    

    
    
    % Colors for the hyperref package
    \definecolor{urlcolor}{rgb}{0,.145,.698}
    \definecolor{linkcolor}{rgb}{.71,0.21,0.01}
    \definecolor{citecolor}{rgb}{.12,.54,.11}

    % ANSI colors
    \definecolor{ansi-black}{HTML}{3E424D}
    \definecolor{ansi-black-intense}{HTML}{282C36}
    \definecolor{ansi-red}{HTML}{E75C58}
    \definecolor{ansi-red-intense}{HTML}{B22B31}
    \definecolor{ansi-green}{HTML}{00A250}
    \definecolor{ansi-green-intense}{HTML}{007427}
    \definecolor{ansi-yellow}{HTML}{DDB62B}
    \definecolor{ansi-yellow-intense}{HTML}{B27D12}
    \definecolor{ansi-blue}{HTML}{208FFB}
    \definecolor{ansi-blue-intense}{HTML}{0065CA}
    \definecolor{ansi-magenta}{HTML}{D160C4}
    \definecolor{ansi-magenta-intense}{HTML}{A03196}
    \definecolor{ansi-cyan}{HTML}{60C6C8}
    \definecolor{ansi-cyan-intense}{HTML}{258F8F}
    \definecolor{ansi-white}{HTML}{C5C1B4}
    \definecolor{ansi-white-intense}{HTML}{A1A6B2}

    % commands and environments needed by pandoc snippets
    % extracted from the output of `pandoc -s`
    \providecommand{\tightlist}{%
      \setlength{\itemsep}{0pt}\setlength{\parskip}{0pt}}
    \DefineVerbatimEnvironment{Highlighting}{Verbatim}{commandchars=\\\{\}}
    % Add ',fontsize=\small' for more characters per line
    \newenvironment{Shaded}{}{}
    \newcommand{\KeywordTok}[1]{\textcolor[rgb]{0.00,0.44,0.13}{\textbf{{#1}}}}
    \newcommand{\DataTypeTok}[1]{\textcolor[rgb]{0.56,0.13,0.00}{{#1}}}
    \newcommand{\DecValTok}[1]{\textcolor[rgb]{0.25,0.63,0.44}{{#1}}}
    \newcommand{\BaseNTok}[1]{\textcolor[rgb]{0.25,0.63,0.44}{{#1}}}
    \newcommand{\FloatTok}[1]{\textcolor[rgb]{0.25,0.63,0.44}{{#1}}}
    \newcommand{\CharTok}[1]{\textcolor[rgb]{0.25,0.44,0.63}{{#1}}}
    \newcommand{\StringTok}[1]{\textcolor[rgb]{0.25,0.44,0.63}{{#1}}}
    \newcommand{\CommentTok}[1]{\textcolor[rgb]{0.38,0.63,0.69}{\textit{{#1}}}}
    \newcommand{\OtherTok}[1]{\textcolor[rgb]{0.00,0.44,0.13}{{#1}}}
    \newcommand{\AlertTok}[1]{\textcolor[rgb]{1.00,0.00,0.00}{\textbf{{#1}}}}
    \newcommand{\FunctionTok}[1]{\textcolor[rgb]{0.02,0.16,0.49}{{#1}}}
    \newcommand{\RegionMarkerTok}[1]{{#1}}
    \newcommand{\ErrorTok}[1]{\textcolor[rgb]{1.00,0.00,0.00}{\textbf{{#1}}}}
    \newcommand{\NormalTok}[1]{{#1}}
    
    % Additional commands for more recent versions of Pandoc
    \newcommand{\ConstantTok}[1]{\textcolor[rgb]{0.53,0.00,0.00}{{#1}}}
    \newcommand{\SpecialCharTok}[1]{\textcolor[rgb]{0.25,0.44,0.63}{{#1}}}
    \newcommand{\VerbatimStringTok}[1]{\textcolor[rgb]{0.25,0.44,0.63}{{#1}}}
    \newcommand{\SpecialStringTok}[1]{\textcolor[rgb]{0.73,0.40,0.53}{{#1}}}
    \newcommand{\ImportTok}[1]{{#1}}
    \newcommand{\DocumentationTok}[1]{\textcolor[rgb]{0.73,0.13,0.13}{\textit{{#1}}}}
    \newcommand{\AnnotationTok}[1]{\textcolor[rgb]{0.38,0.63,0.69}{\textbf{\textit{{#1}}}}}
    \newcommand{\CommentVarTok}[1]{\textcolor[rgb]{0.38,0.63,0.69}{\textbf{\textit{{#1}}}}}
    \newcommand{\VariableTok}[1]{\textcolor[rgb]{0.10,0.09,0.49}{{#1}}}
    \newcommand{\ControlFlowTok}[1]{\textcolor[rgb]{0.00,0.44,0.13}{\textbf{{#1}}}}
    \newcommand{\OperatorTok}[1]{\textcolor[rgb]{0.40,0.40,0.40}{{#1}}}
    \newcommand{\BuiltInTok}[1]{{#1}}
    \newcommand{\ExtensionTok}[1]{{#1}}
    \newcommand{\PreprocessorTok}[1]{\textcolor[rgb]{0.74,0.48,0.00}{{#1}}}
    \newcommand{\AttributeTok}[1]{\textcolor[rgb]{0.49,0.56,0.16}{{#1}}}
    \newcommand{\InformationTok}[1]{\textcolor[rgb]{0.38,0.63,0.69}{\textbf{\textit{{#1}}}}}
    \newcommand{\WarningTok}[1]{\textcolor[rgb]{0.38,0.63,0.69}{\textbf{\textit{{#1}}}}}
    
    
    % Define a nice break command that doesn't care if a line doesn't already
    % exist.
    \def\br{\hspace*{\fill} \\* }
    % Math Jax compatability definitions
    \def\gt{>}
    \def\lt{<}
    % Document parameters
    \title{Metnum-Case4.2}
    
    
    

    % Pygments definitions
    
\makeatletter
\def\PY@reset{\let\PY@it=\relax \let\PY@bf=\relax%
    \let\PY@ul=\relax \let\PY@tc=\relax%
    \let\PY@bc=\relax \let\PY@ff=\relax}
\def\PY@tok#1{\csname PY@tok@#1\endcsname}
\def\PY@toks#1+{\ifx\relax#1\empty\else%
    \PY@tok{#1}\expandafter\PY@toks\fi}
\def\PY@do#1{\PY@bc{\PY@tc{\PY@ul{%
    \PY@it{\PY@bf{\PY@ff{#1}}}}}}}
\def\PY#1#2{\PY@reset\PY@toks#1+\relax+\PY@do{#2}}

\expandafter\def\csname PY@tok@w\endcsname{\def\PY@tc##1{\textcolor[rgb]{0.73,0.73,0.73}{##1}}}
\expandafter\def\csname PY@tok@c\endcsname{\let\PY@it=\textit\def\PY@tc##1{\textcolor[rgb]{0.25,0.50,0.50}{##1}}}
\expandafter\def\csname PY@tok@cp\endcsname{\def\PY@tc##1{\textcolor[rgb]{0.74,0.48,0.00}{##1}}}
\expandafter\def\csname PY@tok@k\endcsname{\let\PY@bf=\textbf\def\PY@tc##1{\textcolor[rgb]{0.00,0.50,0.00}{##1}}}
\expandafter\def\csname PY@tok@kp\endcsname{\def\PY@tc##1{\textcolor[rgb]{0.00,0.50,0.00}{##1}}}
\expandafter\def\csname PY@tok@kt\endcsname{\def\PY@tc##1{\textcolor[rgb]{0.69,0.00,0.25}{##1}}}
\expandafter\def\csname PY@tok@o\endcsname{\def\PY@tc##1{\textcolor[rgb]{0.40,0.40,0.40}{##1}}}
\expandafter\def\csname PY@tok@ow\endcsname{\let\PY@bf=\textbf\def\PY@tc##1{\textcolor[rgb]{0.67,0.13,1.00}{##1}}}
\expandafter\def\csname PY@tok@nb\endcsname{\def\PY@tc##1{\textcolor[rgb]{0.00,0.50,0.00}{##1}}}
\expandafter\def\csname PY@tok@nf\endcsname{\def\PY@tc##1{\textcolor[rgb]{0.00,0.00,1.00}{##1}}}
\expandafter\def\csname PY@tok@nc\endcsname{\let\PY@bf=\textbf\def\PY@tc##1{\textcolor[rgb]{0.00,0.00,1.00}{##1}}}
\expandafter\def\csname PY@tok@nn\endcsname{\let\PY@bf=\textbf\def\PY@tc##1{\textcolor[rgb]{0.00,0.00,1.00}{##1}}}
\expandafter\def\csname PY@tok@ne\endcsname{\let\PY@bf=\textbf\def\PY@tc##1{\textcolor[rgb]{0.82,0.25,0.23}{##1}}}
\expandafter\def\csname PY@tok@nv\endcsname{\def\PY@tc##1{\textcolor[rgb]{0.10,0.09,0.49}{##1}}}
\expandafter\def\csname PY@tok@no\endcsname{\def\PY@tc##1{\textcolor[rgb]{0.53,0.00,0.00}{##1}}}
\expandafter\def\csname PY@tok@nl\endcsname{\def\PY@tc##1{\textcolor[rgb]{0.63,0.63,0.00}{##1}}}
\expandafter\def\csname PY@tok@ni\endcsname{\let\PY@bf=\textbf\def\PY@tc##1{\textcolor[rgb]{0.60,0.60,0.60}{##1}}}
\expandafter\def\csname PY@tok@na\endcsname{\def\PY@tc##1{\textcolor[rgb]{0.49,0.56,0.16}{##1}}}
\expandafter\def\csname PY@tok@nt\endcsname{\let\PY@bf=\textbf\def\PY@tc##1{\textcolor[rgb]{0.00,0.50,0.00}{##1}}}
\expandafter\def\csname PY@tok@nd\endcsname{\def\PY@tc##1{\textcolor[rgb]{0.67,0.13,1.00}{##1}}}
\expandafter\def\csname PY@tok@s\endcsname{\def\PY@tc##1{\textcolor[rgb]{0.73,0.13,0.13}{##1}}}
\expandafter\def\csname PY@tok@sd\endcsname{\let\PY@it=\textit\def\PY@tc##1{\textcolor[rgb]{0.73,0.13,0.13}{##1}}}
\expandafter\def\csname PY@tok@si\endcsname{\let\PY@bf=\textbf\def\PY@tc##1{\textcolor[rgb]{0.73,0.40,0.53}{##1}}}
\expandafter\def\csname PY@tok@se\endcsname{\let\PY@bf=\textbf\def\PY@tc##1{\textcolor[rgb]{0.73,0.40,0.13}{##1}}}
\expandafter\def\csname PY@tok@sr\endcsname{\def\PY@tc##1{\textcolor[rgb]{0.73,0.40,0.53}{##1}}}
\expandafter\def\csname PY@tok@ss\endcsname{\def\PY@tc##1{\textcolor[rgb]{0.10,0.09,0.49}{##1}}}
\expandafter\def\csname PY@tok@sx\endcsname{\def\PY@tc##1{\textcolor[rgb]{0.00,0.50,0.00}{##1}}}
\expandafter\def\csname PY@tok@m\endcsname{\def\PY@tc##1{\textcolor[rgb]{0.40,0.40,0.40}{##1}}}
\expandafter\def\csname PY@tok@gh\endcsname{\let\PY@bf=\textbf\def\PY@tc##1{\textcolor[rgb]{0.00,0.00,0.50}{##1}}}
\expandafter\def\csname PY@tok@gu\endcsname{\let\PY@bf=\textbf\def\PY@tc##1{\textcolor[rgb]{0.50,0.00,0.50}{##1}}}
\expandafter\def\csname PY@tok@gd\endcsname{\def\PY@tc##1{\textcolor[rgb]{0.63,0.00,0.00}{##1}}}
\expandafter\def\csname PY@tok@gi\endcsname{\def\PY@tc##1{\textcolor[rgb]{0.00,0.63,0.00}{##1}}}
\expandafter\def\csname PY@tok@gr\endcsname{\def\PY@tc##1{\textcolor[rgb]{1.00,0.00,0.00}{##1}}}
\expandafter\def\csname PY@tok@ge\endcsname{\let\PY@it=\textit}
\expandafter\def\csname PY@tok@gs\endcsname{\let\PY@bf=\textbf}
\expandafter\def\csname PY@tok@gp\endcsname{\let\PY@bf=\textbf\def\PY@tc##1{\textcolor[rgb]{0.00,0.00,0.50}{##1}}}
\expandafter\def\csname PY@tok@go\endcsname{\def\PY@tc##1{\textcolor[rgb]{0.53,0.53,0.53}{##1}}}
\expandafter\def\csname PY@tok@gt\endcsname{\def\PY@tc##1{\textcolor[rgb]{0.00,0.27,0.87}{##1}}}
\expandafter\def\csname PY@tok@err\endcsname{\def\PY@bc##1{\setlength{\fboxsep}{0pt}\fcolorbox[rgb]{1.00,0.00,0.00}{1,1,1}{\strut ##1}}}
\expandafter\def\csname PY@tok@kc\endcsname{\let\PY@bf=\textbf\def\PY@tc##1{\textcolor[rgb]{0.00,0.50,0.00}{##1}}}
\expandafter\def\csname PY@tok@kd\endcsname{\let\PY@bf=\textbf\def\PY@tc##1{\textcolor[rgb]{0.00,0.50,0.00}{##1}}}
\expandafter\def\csname PY@tok@kn\endcsname{\let\PY@bf=\textbf\def\PY@tc##1{\textcolor[rgb]{0.00,0.50,0.00}{##1}}}
\expandafter\def\csname PY@tok@kr\endcsname{\let\PY@bf=\textbf\def\PY@tc##1{\textcolor[rgb]{0.00,0.50,0.00}{##1}}}
\expandafter\def\csname PY@tok@bp\endcsname{\def\PY@tc##1{\textcolor[rgb]{0.00,0.50,0.00}{##1}}}
\expandafter\def\csname PY@tok@fm\endcsname{\def\PY@tc##1{\textcolor[rgb]{0.00,0.00,1.00}{##1}}}
\expandafter\def\csname PY@tok@vc\endcsname{\def\PY@tc##1{\textcolor[rgb]{0.10,0.09,0.49}{##1}}}
\expandafter\def\csname PY@tok@vg\endcsname{\def\PY@tc##1{\textcolor[rgb]{0.10,0.09,0.49}{##1}}}
\expandafter\def\csname PY@tok@vi\endcsname{\def\PY@tc##1{\textcolor[rgb]{0.10,0.09,0.49}{##1}}}
\expandafter\def\csname PY@tok@vm\endcsname{\def\PY@tc##1{\textcolor[rgb]{0.10,0.09,0.49}{##1}}}
\expandafter\def\csname PY@tok@sa\endcsname{\def\PY@tc##1{\textcolor[rgb]{0.73,0.13,0.13}{##1}}}
\expandafter\def\csname PY@tok@sb\endcsname{\def\PY@tc##1{\textcolor[rgb]{0.73,0.13,0.13}{##1}}}
\expandafter\def\csname PY@tok@sc\endcsname{\def\PY@tc##1{\textcolor[rgb]{0.73,0.13,0.13}{##1}}}
\expandafter\def\csname PY@tok@dl\endcsname{\def\PY@tc##1{\textcolor[rgb]{0.73,0.13,0.13}{##1}}}
\expandafter\def\csname PY@tok@s2\endcsname{\def\PY@tc##1{\textcolor[rgb]{0.73,0.13,0.13}{##1}}}
\expandafter\def\csname PY@tok@sh\endcsname{\def\PY@tc##1{\textcolor[rgb]{0.73,0.13,0.13}{##1}}}
\expandafter\def\csname PY@tok@s1\endcsname{\def\PY@tc##1{\textcolor[rgb]{0.73,0.13,0.13}{##1}}}
\expandafter\def\csname PY@tok@mb\endcsname{\def\PY@tc##1{\textcolor[rgb]{0.40,0.40,0.40}{##1}}}
\expandafter\def\csname PY@tok@mf\endcsname{\def\PY@tc##1{\textcolor[rgb]{0.40,0.40,0.40}{##1}}}
\expandafter\def\csname PY@tok@mh\endcsname{\def\PY@tc##1{\textcolor[rgb]{0.40,0.40,0.40}{##1}}}
\expandafter\def\csname PY@tok@mi\endcsname{\def\PY@tc##1{\textcolor[rgb]{0.40,0.40,0.40}{##1}}}
\expandafter\def\csname PY@tok@il\endcsname{\def\PY@tc##1{\textcolor[rgb]{0.40,0.40,0.40}{##1}}}
\expandafter\def\csname PY@tok@mo\endcsname{\def\PY@tc##1{\textcolor[rgb]{0.40,0.40,0.40}{##1}}}
\expandafter\def\csname PY@tok@ch\endcsname{\let\PY@it=\textit\def\PY@tc##1{\textcolor[rgb]{0.25,0.50,0.50}{##1}}}
\expandafter\def\csname PY@tok@cm\endcsname{\let\PY@it=\textit\def\PY@tc##1{\textcolor[rgb]{0.25,0.50,0.50}{##1}}}
\expandafter\def\csname PY@tok@cpf\endcsname{\let\PY@it=\textit\def\PY@tc##1{\textcolor[rgb]{0.25,0.50,0.50}{##1}}}
\expandafter\def\csname PY@tok@c1\endcsname{\let\PY@it=\textit\def\PY@tc##1{\textcolor[rgb]{0.25,0.50,0.50}{##1}}}
\expandafter\def\csname PY@tok@cs\endcsname{\let\PY@it=\textit\def\PY@tc##1{\textcolor[rgb]{0.25,0.50,0.50}{##1}}}

\def\PYZbs{\char`\\}
\def\PYZus{\char`\_}
\def\PYZob{\char`\{}
\def\PYZcb{\char`\}}
\def\PYZca{\char`\^}
\def\PYZam{\char`\&}
\def\PYZlt{\char`\<}
\def\PYZgt{\char`\>}
\def\PYZsh{\char`\#}
\def\PYZpc{\char`\%}
\def\PYZdl{\char`\$}
\def\PYZhy{\char`\-}
\def\PYZsq{\char`\'}
\def\PYZdq{\char`\"}
\def\PYZti{\char`\~}
% for compatibility with earlier versions
\def\PYZat{@}
\def\PYZlb{[}
\def\PYZrb{]}
\makeatother


    % Exact colors from NB
    \definecolor{incolor}{rgb}{0.0, 0.0, 0.5}
    \definecolor{outcolor}{rgb}{0.545, 0.0, 0.0}



    
    % Prevent overflowing lines due to hard-to-break entities
    \sloppy 
    % Setup hyperref package
    \hypersetup{
      breaklinks=true,  % so long urls are correctly broken across lines
      colorlinks=true,
      urlcolor=urlcolor,
      linkcolor=linkcolor,
      citecolor=citecolor,
      }
    % Slightly bigger margins than the latex defaults
    
    \geometry{verbose,tmargin=1in,bmargin=1in,lmargin=1in,rmargin=1in}
    
    

    \begin{document}
    
    
    \maketitle
    
    

    
    \subsubsection{Example 4.2}\label{example-4.2}

Penyelesaian Example 4.2 dari Buku \textbf{An Introduction to
Computational Fluid Dynamics: The Finite Volume Method} Edisi ke-2 oleh
Henk Kaarle Versteeg \& Weeratunge Malalasekera, Halaman 121.

Diketahui: Tebal plat \(L = 2 cm = 0.02\ m\) dengan konduktivitas
\(k = 0.5\ W/m.K\) dan \emph{Uniform heat generation}
\(q = 1000\ kW/m^{3}\). Pada titik A dan B memiliki suhu sebesar
\(100^{o}C\) dan \(200^{o}C\). Lihat Gambar:

Persamaan Pengatur:
\[\frac{\mathrm{d}}{\mathrm{d}x} \left( k\frac{\mathrm{d}T}{\mathrm{d}x} \right) + q = 0\]

    \section{Solusi}\label{solusi}

Metode ini menggunakan grid sederhana. Domain dibagi menjadi 5 kontrol
volume, sehingga diketahui nilai \(\delta x =\ 0.004\ m\). Grid sebagai
berikut:

\subsection{Noda Tengah (Titik 2, 3, 4)}\label{noda-tengah-titik-2-3-4}

Diintegrasikan persamaan pengatur:
\[ \int\limits_{\Delta V}\ \frac{d}{dx} \left( k\frac{dT}{dx} \right) \text{d}V + \int\limits_{\Delta V} q \text{d}V = 0\]

Suku pertama di integralkan seperti contoh sebelumnya, sedangkan suku
kedua, dievaluasi dengan rata-rata (i.e. \(\bar{S}\Delta V=q\Delta V\))
pada kontrol volume. Persamaan diatas dapat ditulis:

\[\begin{aligned}
\left[ \left( k A \frac{\text{d}T}{\text{d}x} \right)_e -
\left( k A \frac{\text{d}T}{\text{d}x} \right)_w \right] 
+q \Delta V = 0 \\
\left[ k_e A \left(\frac{T_E-T_P}{\delta x} \right) - k_w A \left(\frac{T_P-T_W}{\delta x} \right) \right]
+ q A \delta x = 0 \\
\left(\frac{k_e A}{\delta x} + \frac{k_w A}{\delta x}\right)T_P = \left(\frac{k_w A}{\delta x}\right)T_W +
\left(\frac{k_w A}{\delta x} \right)T_E + q A \delta x \\
\end{aligned}\]

Dalam bentuk umum serupa dengan: \[a_P T_P = a_W T_W+a_E T_E + S_u\]

karena \(k_e = k_w = k\) maka diperoleh persamaan:

\[\begin{array}{|c|c|c|c|c|}\hline a_W & a_E & a_P & S_P & S_u \\
\hline
\frac{k A}{\delta x} & \frac{k A}{\delta x} & a_W+a_E-S_P & 0 & qA\delta x \\ \hline
\end{array}
\]

Persamaan diatas hanya berlaku untuk noda di tengah (contoh: noda 2, 3
dan 4).

\subsection{Noda Tepi Kiri (titik 1)}\label{noda-tepi-kiri-titik-1}

Hasil integrasi persamaan sebelumnya pada titik 1 sebagai berikut:

\[\begin{aligned}
\left[ \left( k A \frac{\text{d}T}{\text{d}x} \right)_e -
\left( k A \frac{\text{d}T}{\text{d}x} \right)_w \right] 
+q \Delta V = 0 \\
\left[ k_e A \left(\frac{T_E-T_P}{\delta x} \right) - k_A A \left(\frac{T_P-T_A}{\delta x/2} \right) \right]
+ q A \delta x = 0 \\
\end{aligned}\]

karena \(k_e = k_w = k\) maka diperoleh persamaan:
\[a_P T_P = a_W T_W+a_E T_E + S_u\]

yang:

\[\begin{array}{|c|c|c|c|c|}\hline a_W & a_E & a_P & S_P & S_u \\
\hline
0 & \frac{k A}{\delta x} & a_W+a_E-S_P & -\frac{2kA}{\delta x} & qA\delta x + \frac{2kA}{\delta x}T_A \\ \hline
\end{array}\]

\subsection{Noda Tepi Kanan (titik 5)}\label{noda-tepi-kanan-titik-5}

Hasil integrasi yang serupa pada titik 5 sebagai berikut:

\[\begin{aligned}
\left[ \left( k A \frac{\text{d}T}{\text{d}x} \right)_e -
\left( k A \frac{\text{d}T}{\text{d}x} \right)_w \right] 
+q \Delta V = 0 \\
\left[ k_B A \left(\frac{T_B-T_P}{\delta x} \right) - k_w A \left(\frac{T_P-T_W}{\delta x/2} \right) \right]
+ q A \delta x = 0 \\
\end{aligned}\]

karena \(k_e = k_w = k\) maka diperoleh persamaan:
\[a_P T_P = a_W T_W+a_E T_E + S_u\]

yang:

\[\begin{array}{|c|c|c|c|c|}\hline a_W & a_E & a_P & S_P & S_u \\
\hline
\frac{k A}{\delta x} & 0 & a_W+a_E-S_P & -\frac{2kA}{\delta x} & qA\delta x + \frac{2kA}{\delta x}T_B \\ \hline
\end{array}\]

\begin{center}\rule{0.5\linewidth}{\linethickness}\end{center}

\subsection{Dalam Bentuk Matriks}\label{dalam-bentuk-matriks}

Persamaan diatas dapat disusun dalam bentuk matriks (jika 5 noda):

\[\left[ \begin{array}{ccccc}a_P & -a_E & 0 & 0 & 0 \\
-a_W & a_P & -a_E & 0 & 0 \\
0 & -a_W & a_P & -a_E & 0 \\
0 & 0 & -a_W & a_P & -a_E \\
0 & 0 & 0 & -a_W & a_P \\\end{array}\right]
\left[ \begin{array}{c} T_1 \\ T_2 \\ T_3 \\ T_4 \\ T_5 \end{array} \right] = 
\left[ \begin{array}{c} S_u \\ S_u \\ S_u \\ S_u \\ S_u \end{array} \right]
\]

dengan catatan, pada baris \(1\) dan \(5\) nilai \(a_W, a_P, a_E\)
menggunakan persamaan yang berbeda dari baris lainnya. Matrix
\(\left[ \begin{array}{c} T_1 \\ T_2 \\ T_3 \\ T_4 \\ T_5 \end{array} \right]\)
dapat dicari dengan menggunakan metoda algoritma Thomas. Dalam python
dapat menggunakan \emph{numpy.linalg.solve()} dari library \emph{numpy}.

\subsection{Perbandingan solusi
analitik}\label{perbandingan-solusi-analitik}

Hasil numerik bisa dibandingkan dengan persamaan solusi analitik:
\[T = \left[\frac{T_B-T_A}{L}+\frac{q}{2k}(L-x)\right]x+T_A\]

    \section{Pemrograman Python}\label{pemrograman-python}

Solusi numerik dicoba menggunakan Python dengan menggunakan beberapa
library antara lain numpy, decimal, matplotlib. Menggunakan fungsi
\emph{add\_dec} dan \emph{create\_axis} yang diperoleh dari
\textbf{metnum\_uma.py}.

    \begin{Verbatim}[commandchars=\\\{\}]
{\color{incolor}In [{\color{incolor}4}]:} \PY{k+kn}{import} \PY{n+nn}{numpy} \PY{k}{as} \PY{n+nn}{np}
        \PY{k+kn}{from} \PY{n+nn}{decimal} \PY{k}{import} \PY{n}{Decimal} \PY{k}{as} \PY{n}{dec}
        \PY{k+kn}{from} \PY{n+nn}{metnum\PYZus{}uma} \PY{k}{import} \PY{n}{add\PYZus{}dec}\PY{p}{,} \PY{n}{create\PYZus{}axis} 
        
        \PY{c+c1}{\PYZsh{} Diketahui:}
        \PY{n}{L} \PY{o}{=} \PY{l+m+mf}{0.02}  \PY{c+c1}{\PYZsh{} (m) \PYZhy{} Tebal plat}
        \PY{n}{k} \PY{o}{=} \PY{l+m+mf}{0.5}  \PY{c+c1}{\PYZsh{} (W/m.K) \PYZhy{} konduktivitas}
        \PY{n}{q} \PY{o}{=} \PY{l+m+mi}{1000}\PY{o}{*}\PY{l+m+mi}{1000}  \PY{c+c1}{\PYZsh{} (kW/m\PYZca{}3) \PYZhy{} Uniform Heat Generation}
        \PY{n}{TA}\PY{p}{,} \PY{n}{TB} \PY{o}{=} \PY{l+m+mi}{100}\PY{p}{,} \PY{l+m+mi}{200}  \PY{c+c1}{\PYZsh{} (C) \PYZhy{} suhu di titik A}
        \PY{n}{dx} \PY{o}{=} \PY{l+m+mf}{0.002}  \PY{c+c1}{\PYZsh{} (m) \PYZhy{} jarak grid control volume}
        \PY{n}{A} \PY{o}{=} \PY{l+m+mi}{1}  \PY{c+c1}{\PYZsh{} (m\PYZca{}2) \PYZhy{} Luasan}
        
        \PY{n}{nodes} \PY{o}{=} \PY{n+nb}{int}\PY{p}{(}\PY{n}{L}\PY{o}{/}\PY{n}{dx}\PY{p}{)}
        \PY{n+nb}{print}\PY{p}{(}\PY{l+s+s1}{\PYZsq{}}\PY{l+s+s1}{Suhu di titik }\PY{l+s+se}{\PYZbs{}t}\PY{l+s+s1}{A = }\PY{l+s+si}{\PYZob{}:\PYZgt{}5d\PYZcb{}}\PY{l+s+s1}{ C}\PY{l+s+se}{\PYZbs{}n}\PY{l+s+s1}{Suhu di titik }\PY{l+s+se}{\PYZbs{}t}\PY{l+s+s1}{B = }\PY{l+s+si}{\PYZob{}:\PYZgt{}5d\PYZcb{}}\PY{l+s+s1}{ C}\PY{l+s+se}{\PYZbs{}n}\PY{l+s+s1}{dengan panjang }\PY{l+s+se}{\PYZbs{}t}\PY{l+s+s1}{L = }\PY{l+s+si}{\PYZob{}:\PYZgt{}1.3f\PYZcb{}}\PY{l+s+s1}{ m}\PY{l+s+s1}{\PYZsq{}}\PY{o}{.}\PY{n}{format}\PY{p}{(}
            \PY{n}{TA}\PY{p}{,} \PY{n}{TB}\PY{p}{,} \PY{n}{L}\PY{p}{)}\PY{p}{)}
        \PY{n+nb}{print}\PY{p}{(}\PY{l+s+s1}{\PYZsq{}}\PY{l+s+s1}{Jumlah Noda }\PY{l+s+se}{\PYZbs{}t}\PY{l+s+s1}{  = }\PY{l+s+si}{\PYZob{}:\PYZgt{}5d\PYZcb{}}\PY{l+s+se}{\PYZbs{}t}\PY{l+s+s1}{dengan dx = }\PY{l+s+si}{\PYZob{}:1.3f\PYZcb{}}\PY{l+s+s1}{ m}\PY{l+s+s1}{\PYZsq{}}\PY{o}{.}\PY{n}{format}\PY{p}{(}\PY{n}{nodes}\PY{p}{,} \PY{n}{dx}\PY{p}{)}\PY{p}{)}
        \PY{n}{axisx} \PY{o}{=} \PY{n}{create\PYZus{}axis}\PY{p}{(}\PY{n}{nodes}\PY{p}{,} \PY{n}{dx}\PY{p}{)}
\end{Verbatim}


    \begin{Verbatim}[commandchars=\\\{\}]
Suhu di titik 	A =   100 C
Suhu di titik 	B =   200 C
dengan panjang 	L = 0.020 m
Jumlah Noda 	  =    10	dengan dx = 0.002 m

    \end{Verbatim}

    \begin{Verbatim}[commandchars=\\\{\}]
{\color{incolor}In [{\color{incolor}5}]:} \PY{c+c1}{\PYZsh{} Buat Matrix Penyelesaian}
        \PY{n}{mat\PYZus{}a} \PY{o}{=} \PY{n}{np}\PY{o}{.}\PY{n}{zeros}\PY{p}{(}\PY{p}{[}\PY{n}{nodes}\PY{p}{,} \PY{n}{nodes}\PY{p}{]}\PY{p}{)}
        \PY{n}{mat\PYZus{}d} \PY{o}{=} \PY{n}{np}\PY{o}{.}\PY{n}{zeros}\PY{p}{(}\PY{p}{[}\PY{n}{nodes}\PY{p}{]}\PY{p}{)}
        
        \PY{c+c1}{\PYZsh{} Buat Matrix A}
        \PY{k}{for} \PY{n}{i} \PY{o+ow}{in} \PY{n+nb}{range}\PY{p}{(}\PY{l+m+mi}{0}\PY{p}{,} \PY{n}{nodes}\PY{p}{)}\PY{p}{:}
            \PY{k}{for} \PY{n}{j} \PY{o+ow}{in} \PY{n+nb}{range}\PY{p}{(}\PY{l+m+mi}{0}\PY{p}{,} \PY{n}{nodes}\PY{p}{)}\PY{p}{:}
                \PY{k}{if} \PY{n}{i} \PY{o}{==} \PY{n}{j} \PY{o+ow}{and} \PY{p}{(}\PY{n}{i} \PY{o}{==} \PY{l+m+mi}{0}\PY{p}{)}\PY{p}{:}  \PY{c+c1}{\PYZsh{} Baris Pertama (Titik A)}
                    \PY{n}{aW} \PY{o}{=} \PY{l+m+mi}{0}
                    \PY{n}{aE} \PY{o}{=} \PY{n}{k}\PY{o}{*}\PY{n}{A}\PY{o}{/}\PY{n}{dx}
                    \PY{n}{SP} \PY{o}{=} \PY{o}{\PYZhy{}}\PY{l+m+mi}{2}\PY{o}{*}\PY{n}{k}\PY{o}{*}\PY{n}{A}\PY{o}{/}\PY{n}{dx}
                    \PY{n}{aP} \PY{o}{=} \PY{n}{aW} \PY{o}{+} \PY{n}{aE} \PY{o}{\PYZhy{}} \PY{n}{SP}
                    \PY{n}{mat\PYZus{}a}\PY{p}{[}\PY{n}{i}\PY{p}{,} \PY{n}{j}\PY{p}{]} \PY{o}{=} \PY{n}{aP}
                    \PY{n}{mat\PYZus{}a}\PY{p}{[}\PY{n}{i}\PY{p}{,} \PY{n}{j}\PY{o}{+}\PY{l+m+mi}{1}\PY{p}{]} \PY{o}{=} \PY{o}{\PYZhy{}}\PY{n}{aE}
                \PY{c+c1}{\PYZsh{} Baris Kedua sampai satu sebelum terakhir}
                \PY{k}{elif} \PY{n}{i} \PY{o}{==} \PY{n}{j} \PY{o+ow}{and} \PY{p}{(}\PY{n}{i} \PY{o}{\PYZgt{}} \PY{l+m+mi}{0} \PY{o+ow}{and} \PY{n}{i} \PY{o}{\PYZlt{}} \PY{n}{nodes}\PY{o}{\PYZhy{}}\PY{l+m+mi}{1}\PY{p}{)}\PY{p}{:}
                    \PY{n}{aW} \PY{o}{=} \PY{n}{k}\PY{o}{*}\PY{n}{A}\PY{o}{/}\PY{n}{dx}
                    \PY{n}{aE} \PY{o}{=} \PY{n}{k}\PY{o}{*}\PY{n}{A}\PY{o}{/}\PY{n}{dx}
                    \PY{n}{SP} \PY{o}{=} \PY{l+m+mi}{0}
                    \PY{n}{aP} \PY{o}{=} \PY{n}{aW} \PY{o}{+} \PY{n}{aE} \PY{o}{\PYZhy{}} \PY{n}{SP}
                    \PY{n}{mat\PYZus{}a}\PY{p}{[}\PY{n}{i}\PY{p}{,} \PY{n}{j}\PY{o}{\PYZhy{}}\PY{l+m+mi}{1}\PY{p}{]} \PY{o}{=} \PY{o}{\PYZhy{}}\PY{n}{aW}
                    \PY{n}{mat\PYZus{}a}\PY{p}{[}\PY{n}{i}\PY{p}{,} \PY{n}{j}\PY{p}{]} \PY{o}{=} \PY{n}{aP}
                    \PY{n}{mat\PYZus{}a}\PY{p}{[}\PY{n}{i}\PY{p}{,} \PY{n}{j}\PY{o}{+}\PY{l+m+mi}{1}\PY{p}{]} \PY{o}{=} \PY{o}{\PYZhy{}}\PY{n}{aE}
                \PY{k}{elif} \PY{n}{i} \PY{o}{==} \PY{n}{j} \PY{o+ow}{and} \PY{p}{(}\PY{n}{i} \PY{o}{==} \PY{n}{nodes}\PY{o}{\PYZhy{}}\PY{l+m+mi}{1}\PY{p}{)}\PY{p}{:}  \PY{c+c1}{\PYZsh{} Baris Terakhir (Titik B)}
                    \PY{n}{aW} \PY{o}{=} \PY{n}{k}\PY{o}{*}\PY{n}{A}\PY{o}{/}\PY{n}{dx}
                    \PY{n}{aE} \PY{o}{=} \PY{l+m+mi}{0}
                    \PY{n}{SP} \PY{o}{=} \PY{o}{\PYZhy{}}\PY{l+m+mi}{2}\PY{o}{*}\PY{n}{k}\PY{o}{*}\PY{n}{A}\PY{o}{/}\PY{n}{dx}
                    \PY{n}{aP} \PY{o}{=} \PY{n}{aW} \PY{o}{+} \PY{n}{aE} \PY{o}{\PYZhy{}} \PY{n}{SP}
                    \PY{n}{mat\PYZus{}a}\PY{p}{[}\PY{n}{i}\PY{p}{,} \PY{n}{j}\PY{o}{\PYZhy{}}\PY{l+m+mi}{1}\PY{p}{]} \PY{o}{=} \PY{o}{\PYZhy{}}\PY{n}{aW}
                    \PY{n}{mat\PYZus{}a}\PY{p}{[}\PY{n}{i}\PY{p}{,} \PY{n}{j}\PY{p}{]} \PY{o}{=} \PY{n}{aP}
        
        \PY{c+c1}{\PYZsh{} Matrix D}
        \PY{k}{for} \PY{n}{i} \PY{o+ow}{in} \PY{n+nb}{range}\PY{p}{(}\PY{l+m+mi}{0}\PY{p}{,} \PY{n}{nodes}\PY{p}{)}\PY{p}{:}
            \PY{k}{if} \PY{n}{i} \PY{o}{==} \PY{l+m+mi}{0}\PY{p}{:}
                \PY{n}{Su} \PY{o}{=} \PY{l+m+mi}{2}\PY{o}{*}\PY{n}{k}\PY{o}{*}\PY{n}{A}\PY{o}{*}\PY{n}{TA}\PY{o}{/}\PY{n}{dx} \PY{o}{+} \PY{n}{q}\PY{o}{*}\PY{n}{A}\PY{o}{*}\PY{n}{dx}
                \PY{n}{mat\PYZus{}d}\PY{p}{[}\PY{n}{i}\PY{p}{]} \PY{o}{=} \PY{n}{Su}
            \PY{k}{elif} \PY{n}{i} \PY{o}{\PYZgt{}} \PY{l+m+mi}{0} \PY{o+ow}{and} \PY{n}{i} \PY{o}{\PYZlt{}} \PY{n}{nodes}\PY{o}{\PYZhy{}}\PY{l+m+mi}{1}\PY{p}{:}
                \PY{n}{Su} \PY{o}{=} \PY{n}{q}\PY{o}{*}\PY{n}{A}\PY{o}{*}\PY{n}{dx}
                \PY{n}{mat\PYZus{}d}\PY{p}{[}\PY{n}{i}\PY{p}{]} \PY{o}{=} \PY{n}{Su}
            \PY{k}{elif} \PY{n}{i} \PY{o}{==} \PY{n}{nodes}\PY{o}{\PYZhy{}}\PY{l+m+mi}{1}\PY{p}{:}
                \PY{n}{Su} \PY{o}{=} \PY{n}{q}\PY{o}{*}\PY{n}{A}\PY{o}{*}\PY{n}{dx} \PY{o}{+} \PY{l+m+mi}{2}\PY{o}{*}\PY{n}{k}\PY{o}{*}\PY{n}{A}\PY{o}{*}\PY{n}{TB}\PY{o}{/}\PY{n}{dx}
                \PY{n}{mat\PYZus{}d}\PY{p}{[}\PY{n}{i}\PY{p}{]} \PY{o}{=} \PY{n}{Su}
        
        \PY{c+c1}{\PYZsh{} Penyelesaian Matrix}
        \PY{n}{result} \PY{o}{=} \PY{n}{np}\PY{o}{.}\PY{n}{linalg}\PY{o}{.}\PY{n}{solve}\PY{p}{(}\PY{n}{mat\PYZus{}a}\PY{p}{,} \PY{n}{mat\PYZus{}d}\PY{p}{)}
        \PY{n}{y\PYZus{}num} \PY{o}{=} \PY{n}{np}\PY{o}{.}\PY{n}{append}\PY{p}{(}\PY{n}{result}\PY{p}{,} \PY{n}{TB}\PY{p}{)}
        \PY{n}{y\PYZus{}num} \PY{o}{=} \PY{n}{np}\PY{o}{.}\PY{n}{insert}\PY{p}{(}\PY{n}{y\PYZus{}num}\PY{p}{,} \PY{l+m+mi}{0}\PY{p}{,} \PY{n}{TA}\PY{p}{)}
        
        \PY{k}{def} \PY{n+nf}{hasil\PYZus{}exact}\PY{p}{(}\PY{n}{x}\PY{p}{,} \PY{n}{TB}\PY{p}{,} \PY{n}{TA}\PY{p}{,} \PY{n}{L}\PY{p}{,} \PY{n}{q}\PY{p}{,} \PY{n}{k}\PY{p}{)}\PY{p}{:}
            \PY{k}{return} \PY{p}{(}\PY{p}{(}\PY{n}{TB}\PY{o}{\PYZhy{}}\PY{n}{TA}\PY{p}{)}\PY{o}{/}\PY{n}{L} \PY{o}{+} \PY{p}{(}\PY{n}{q}\PY{o}{/}\PY{p}{(}\PY{l+m+mi}{2}\PY{o}{*}\PY{n}{k}\PY{p}{)}\PY{o}{*}\PY{p}{(}\PY{n}{L}\PY{o}{\PYZhy{}}\PY{n}{x}\PY{p}{)}\PY{p}{)}\PY{p}{)}\PY{o}{*}\PY{n}{x} \PY{o}{+} \PY{n}{TA}
        
        \PY{c+c1}{\PYZsh{} Solusi Analitik}
        \PY{n}{nodes\PYZus{}exact} \PY{o}{=} \PY{l+m+mi}{100}  \PY{c+c1}{\PYZsh{} Jumlah titik untuk solusi analitik}
        \PY{n}{exact} \PY{o}{=} \PY{p}{[}\PY{p}{]}
        \PY{n}{axisx\PYZus{}exact} \PY{o}{=} \PY{n}{np}\PY{o}{.}\PY{n}{linspace}\PY{p}{(}\PY{l+m+mi}{0}\PY{p}{,} \PY{n}{L}\PY{p}{,} \PY{n}{nodes\PYZus{}exact}\PY{p}{)}
        \PY{k}{for} \PY{n}{x} \PY{o+ow}{in} \PY{n}{axisx\PYZus{}exact}\PY{p}{:}
            \PY{c+c1}{\PYZsh{}hasil = ((TB\PYZhy{}TA)/L + (q/(2*k)*(L\PYZhy{}x)))*x + TA}
            \PY{n}{hasil} \PY{o}{=} \PY{n}{hasil\PYZus{}exact}\PY{p}{(}\PY{n}{x}\PY{p}{,} \PY{n}{TB}\PY{p}{,} \PY{n}{TA}\PY{p}{,} \PY{n}{L}\PY{p}{,} \PY{n}{q}\PY{p}{,} \PY{n}{k}\PY{p}{)}
            \PY{n}{exact}\PY{o}{.}\PY{n}{append}\PY{p}{(}\PY{n}{hasil}\PY{p}{)}
        
        \PY{n+nb}{print}\PY{p}{(}\PY{l+s+s1}{\PYZsq{}}\PY{l+s+s1}{Matrix A = }\PY{l+s+se}{\PYZbs{}n}\PY{l+s+si}{\PYZob{}\PYZcb{}}\PY{l+s+se}{\PYZbs{}n}\PY{l+s+s1}{=====}\PY{l+s+s1}{\PYZsq{}}\PY{o}{.}\PY{n}{format}\PY{p}{(}\PY{n}{mat\PYZus{}a}\PY{p}{)}\PY{p}{)}
        \PY{n+nb}{print}\PY{p}{(}\PY{l+s+s1}{\PYZsq{}}\PY{l+s+s1}{Matrix D = }\PY{l+s+se}{\PYZbs{}n}\PY{l+s+si}{\PYZob{}\PYZcb{}}\PY{l+s+se}{\PYZbs{}n}\PY{l+s+s1}{=====}\PY{l+s+s1}{\PYZsq{}}\PY{o}{.}\PY{n}{format}\PY{p}{(}\PY{n}{mat\PYZus{}d}\PY{p}{)}\PY{p}{)}
        \PY{n+nb}{print}\PY{p}{(}
            \PY{l+s+s1}{\PYZsq{}}\PY{l+s+s1}{Penyelesaian Matrix [AX = D], diperoleh matrix X = }\PY{l+s+se}{\PYZbs{}n}\PY{l+s+si}{\PYZob{}\PYZcb{}}\PY{l+s+se}{\PYZbs{}n}\PY{l+s+s1}{=====}\PY{l+s+s1}{\PYZsq{}}\PY{o}{.}\PY{n}{format}\PY{p}{(}\PY{n}{result}\PY{p}{)}\PY{p}{)}
\end{Verbatim}


    \begin{Verbatim}[commandchars=\\\{\}]
Matrix A = 
[[ 750. -250.    0.    0.    0.    0.    0.    0.    0.    0.]
 [-250.  500. -250.    0.    0.    0.    0.    0.    0.    0.]
 [   0. -250.  500. -250.    0.    0.    0.    0.    0.    0.]
 [   0.    0. -250.  500. -250.    0.    0.    0.    0.    0.]
 [   0.    0.    0. -250.  500. -250.    0.    0.    0.    0.]
 [   0.    0.    0.    0. -250.  500. -250.    0.    0.    0.]
 [   0.    0.    0.    0.    0. -250.  500. -250.    0.    0.]
 [   0.    0.    0.    0.    0.    0. -250.  500. -250.    0.]
 [   0.    0.    0.    0.    0.    0.    0. -250.  500. -250.]
 [   0.    0.    0.    0.    0.    0.    0.    0. -250.  750.]]
=====
Matrix D = 
[ 52000.   2000.   2000.   2000.   2000.   2000.   2000.   2000.   2000.
 102000.]
=====
Penyelesaian Matrix [AX = D], diperoleh matrix X = 
[125. 167. 201. 227. 245. 255. 257. 251. 237. 215.]
=====

    \end{Verbatim}

    \begin{Verbatim}[commandchars=\\\{\}]
{\color{incolor}In [{\color{incolor}6}]:} \PY{k+kn}{import} \PY{n+nn}{matplotlib}\PY{n+nn}{.}\PY{n+nn}{pyplot} \PY{k}{as} \PY{n+nn}{plt}
        \PY{o}{\PYZpc{}}\PY{k}{matplotlib} inline
        
        \PY{n}{plt}\PY{o}{.}\PY{n}{rcParams}\PY{p}{[}\PY{l+s+s2}{\PYZdq{}}\PY{l+s+s2}{figure.figsize}\PY{l+s+s2}{\PYZdq{}}\PY{p}{]} \PY{o}{=} \PY{p}{(}\PY{l+m+mi}{15}\PY{p}{,} \PY{l+m+mi}{8}\PY{p}{)}
        \PY{n}{plt}\PY{o}{.}\PY{n}{title}\PY{p}{(}\PY{l+s+s1}{\PYZsq{}}\PY{l+s+s1}{Comparison Numerical vs. Analytical Solution}\PY{l+s+s1}{\PYZsq{}}\PY{p}{)}
        \PY{n}{plt}\PY{o}{.}\PY{n}{xlabel}\PY{p}{(}\PY{l+s+s1}{\PYZsq{}}\PY{l+s+s1}{Distance x (m)}\PY{l+s+s1}{\PYZsq{}}\PY{p}{)}
        \PY{n}{plt}\PY{o}{.}\PY{n}{ylabel}\PY{p}{(}\PY{l+s+s1}{\PYZsq{}}\PY{l+s+s1}{Temperature \PYZdl{}(\PYZca{}}\PY{l+s+si}{\PYZob{}o\PYZcb{}}\PY{l+s+s1}{C)\PYZdl{}}\PY{l+s+s1}{\PYZsq{}}\PY{p}{)}
        \PY{n}{plt}\PY{o}{.}\PY{n}{plot}\PY{p}{(}\PY{n}{axisx}\PY{p}{,} \PY{n}{y\PYZus{}num}\PY{p}{,} \PY{l+s+s1}{\PYZsq{}}\PY{l+s+s1}{ro}\PY{l+s+s1}{\PYZsq{}}\PY{p}{,} \PY{n}{label}\PY{o}{=}\PY{l+s+s1}{\PYZsq{}}\PY{l+s+s1}{numerical}\PY{l+s+s1}{\PYZsq{}}\PY{p}{)}
        \PY{n}{plt}\PY{o}{.}\PY{n}{plot}\PY{p}{(}\PY{n}{axisx\PYZus{}exact}\PY{p}{,} \PY{n}{exact}\PY{p}{,} \PY{l+s+s1}{\PYZsq{}}\PY{l+s+s1}{y\PYZhy{}\PYZhy{}}\PY{l+s+s1}{\PYZsq{}}\PY{p}{,} \PY{n}{label}\PY{o}{=}\PY{l+s+s1}{\PYZsq{}}\PY{l+s+s1}{analytical}\PY{l+s+s1}{\PYZsq{}}\PY{p}{)}
        \PY{n}{plt}\PY{o}{.}\PY{n}{legend}\PY{p}{(}\PY{p}{)}
        
        \PY{n}{node\PYZus{}ann} \PY{o}{=} \PY{l+m+mi}{3}
        \PY{n}{plt}\PY{o}{.}\PY{n}{annotate}\PY{p}{(}\PY{l+s+s1}{\PYZsq{}}\PY{l+s+s1}{Numerical}\PY{l+s+s1}{\PYZsq{}}\PY{p}{,} \PY{n}{xy}\PY{o}{=}\PY{p}{(}\PY{n}{axisx}\PY{p}{[}\PY{n}{node\PYZus{}ann}\PY{p}{]}\PY{p}{,} \PY{n}{y\PYZus{}num}\PY{p}{[}\PY{n}{node\PYZus{}ann}\PY{p}{]}\PY{p}{)}\PY{p}{,} \PY{n}{xytext}\PY{o}{=}\PY{p}{(}\PY{n}{axisx}\PY{p}{[}\PY{n}{node\PYZus{}ann}\PY{p}{]}\PY{p}{,} \PY{n}{y\PYZus{}num}\PY{p}{[}\PY{n}{node\PYZus{}ann}\PY{p}{]}\PY{o}{\PYZhy{}}\PY{l+m+mi}{20}\PY{p}{)}\PY{p}{,}
                     \PY{n}{arrowprops}\PY{o}{=}\PY{n+nb}{dict}\PY{p}{(}\PY{n}{arrowstyle}\PY{o}{=}\PY{l+s+s1}{\PYZsq{}}\PY{l+s+s1}{\PYZhy{}\PYZgt{}}\PY{l+s+s1}{\PYZsq{}}\PY{p}{,} \PY{n}{connectionstyle}\PY{o}{=}\PY{l+s+s1}{\PYZsq{}}\PY{l+s+s1}{arc3}\PY{l+s+s1}{\PYZsq{}}\PY{p}{)}\PY{p}{,}\PY{p}{)}
        
        \PY{n}{mid\PYZus{}exact} \PY{o}{=} \PY{n+nb}{int}\PY{p}{(}\PY{n}{nodes\PYZus{}exact}\PY{o}{/}\PY{l+m+mi}{2}\PY{p}{)}\PY{o}{\PYZhy{}}\PY{l+m+mi}{10}
        \PY{n}{plt}\PY{o}{.}\PY{n}{annotate}\PY{p}{(}\PY{l+s+s1}{\PYZsq{}}\PY{l+s+s1}{Exact Solution}\PY{l+s+s1}{\PYZsq{}}\PY{p}{,} \PY{n}{xy}\PY{o}{=}\PY{p}{(}\PY{n}{axisx\PYZus{}exact}\PY{p}{[}\PY{n}{mid\PYZus{}exact}\PY{p}{]}\PY{p}{,} \PY{n}{exact}\PY{p}{[}\PY{n}{mid\PYZus{}exact}\PY{p}{]}\PY{p}{)}\PY{p}{,}
                     \PY{n}{xytext}\PY{o}{=}\PY{p}{(}\PY{n}{axisx\PYZus{}exact}\PY{p}{[}\PY{n}{mid\PYZus{}exact}\PY{p}{]}\PY{p}{,} \PY{n}{exact}\PY{p}{[}\PY{n}{mid\PYZus{}exact}\PY{p}{]}\PY{o}{\PYZhy{}}\PY{l+m+mi}{20}\PY{p}{)}\PY{p}{,}
                     \PY{n}{arrowprops}\PY{o}{=}\PY{n+nb}{dict}\PY{p}{(}\PY{n}{arrowstyle}\PY{o}{=}\PY{l+s+s1}{\PYZsq{}}\PY{l+s+s1}{\PYZhy{}\PYZgt{}}\PY{l+s+s1}{\PYZsq{}}\PY{p}{,} \PY{n}{connectionstyle}\PY{o}{=}\PY{l+s+s1}{\PYZsq{}}\PY{l+s+s1}{arc3}\PY{l+s+s1}{\PYZsq{}}\PY{p}{)}\PY{p}{,}\PY{p}{)}
        \PY{n}{plt}\PY{o}{.}\PY{n}{grid}\PY{p}{(}\PY{p}{)}
        \PY{n}{plt}\PY{o}{.}\PY{n}{show}\PY{p}{(}\PY{p}{)}
        
        \PY{n}{result\PYZus{}exact} \PY{o}{=} \PY{p}{[}\PY{p}{]}
        \PY{k}{for} \PY{n}{x} \PY{o+ow}{in} \PY{n}{axisx}\PY{p}{:}
            \PY{n}{result\PYZus{}exact}\PY{o}{.}\PY{n}{append}\PY{p}{(}\PY{n}{hasil\PYZus{}exact}\PY{p}{(}\PY{n}{x}\PY{p}{,} \PY{n}{TB}\PY{p}{,} \PY{n}{TA}\PY{p}{,} \PY{n}{L}\PY{p}{,} \PY{n}{q}\PY{p}{,} \PY{n}{k}\PY{p}{)}\PY{p}{)}
        
        \PY{k}{for} \PY{n}{counter}\PY{p}{,} \PY{n}{val} \PY{o+ow}{in} \PY{n+nb}{enumerate}\PY{p}{(}\PY{n}{result\PYZus{}exact}\PY{p}{)}\PY{p}{:}
            \PY{n}{hasil} \PY{o}{=} \PY{p}{(}\PY{n}{y\PYZus{}num}\PY{p}{[}\PY{n}{counter}\PY{p}{]} \PY{o}{\PYZhy{}} \PY{n}{val}\PY{p}{)}\PY{o}{/}\PY{n}{val}\PY{o}{*}\PY{l+m+mi}{100}
            \PY{n+nb}{print}\PY{p}{(}\PY{l+s+s1}{\PYZsq{}}\PY{l+s+s1}{Percentage Error pada titik }\PY{l+s+si}{\PYZob{}:3d\PYZcb{}}\PY{l+s+s1}{ = }\PY{l+s+si}{\PYZob{}:2.2f\PYZcb{}}\PY{l+s+s1}{\PYZsq{}}\PY{o}{.}\PY{n}{format}\PY{p}{(}\PY{n}{counter}\PY{p}{,} \PY{n}{hasil}\PY{p}{)}\PY{p}{)}
\end{Verbatim}


    \begin{center}
    \adjustimage{max size={0.9\linewidth}{0.9\paperheight}}{output_5_0.png}
    \end{center}
    { \hspace*{\fill} \\}
    
    \begin{Verbatim}[commandchars=\\\{\}]
Percentage Error pada titik   0 = 0.00
Percentage Error pada titik   1 = 0.81
Percentage Error pada titik   2 = 0.60
Percentage Error pada titik   3 = 0.50
Percentage Error pada titik   4 = 0.44
Percentage Error pada titik   5 = 0.41
Percentage Error pada titik   6 = 0.39
Percentage Error pada titik   7 = 0.39
Percentage Error pada titik   8 = 0.40
Percentage Error pada titik   9 = 0.42
Percentage Error pada titik  10 = 0.47
Percentage Error pada titik  11 = 0.00

    \end{Verbatim}


    % Add a bibliography block to the postdoc
    
    
    
    \end{document}
